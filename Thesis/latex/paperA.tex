%%%%%%%%%%%%%%%%%%%%%%%%%%%%%%%%%%%%%%%%%%%%%%%%%%%%%%%%%%%%%%%%%%%%%%%%%%%%%%%%%%%%%%%%%%%
\section{The Whitham equation as a model for surface water waves}
%%%%%%%%%%%%%%%%%%%%%%%%%%%%%%%%%%%%%%%%%%%%%%%%%%%%%%%%%%%%%%%%%%%%%%%%%%%%%%%%%%%%%%%%%%%
\subsection{Introduction.}
The Whitham equation was proposed as an alternate model equation for the simplified description of unidirectional 
wave motion at the surface of an inviscid fluid. As the Whitham equation incorporates the full
linear dispersion relation of the water wave problem, it is thought to provide a more faithful description
of shorter waves of small amplitude than traditional long wave models such as the KdV equation.
In this work, we identify a scaling regime in which the Whitham equation can be derived from the
Hamiltonian theory of surface water waves. A Hamiltonian system of Whitham type allowing for two-
way wave propagation is also derived. The Whitham equation is integrated numerically, and it is shown
that the equation gives a close approximation of inviscid free surface dynamics as described by the
Euler equations. The performance of the Whitham equation as a model for free surface dynamics is also
compared to different free surface models: the KdV equation, the BBM equation, and the Pad\'e (2,2) model.
It is found that in a wide parameter range of amplitudes and wavelengths, the Whitham equation performs
on par with or better than the three considered models.


In its simplest form, the water-wave problem concerns the
flow of an incompressible inviscid fluid with a free surface over
a horizontal impenetrable bed. In this situation, the fluid flow
is described by the Euler equations with appropriate boundary
conditions, and the dynamics of the free surface are of particular
interest in the solution of this problem.


There are a number of model equations which allow the
approximate description of the evolution of the free surface
without having to provide a complete solution of the fluid flow
below the surface. In the present contribution, interest is focused
on the derivation and evaluation of a nonlocal water-wave model
known as the Whitham equation. The equation is written as
%
\begin{equation}
	\eta_t + \frac{3}{2} \frac{c_0}{h_0} \eta \eta_x + K_{h_0} \ast \eta_x = 0, 
	\label{whitham-equation}
\end{equation}
%
where the convolution kernel $K_{h_0}$ is given in terms of the Fourier
transform by
%
\begin{equation}
	\mathcal{F}K_{h_0} (\xi) =  \sqrt{\frac{g \tanh(h_0\xi)}{\xi}}.
	\label{whitham-kernel}
\end{equation}
%
$g$ is the gravitational acceleration, $h_0$ is the undisturbed depth of
the fluid, and $c_0 = \sqrt{gh_0}$ is the corresponding long-wave speed.
The convolution can be thought of as a Fourier multiplier operator,
and \eqref{whitham-kernel} represents the Fourier symbol of the operator.


The Whitham equation was proposed by Whitham \cite{Whitham1967} as an
alternative to the well known Korteweg–de Vries (KdV) equation
%
\begin{equation}
	\eta_t + c_0 \eta_x + \frac{3}{2} \frac{c_0}{h_0} \eta \eta_x + \frac{1}{6} c_0 h_0^2 		\eta_{xxx} = 0.
	\label{kdv-equation}
\end{equation}
%
The validity of the KdV equation as a model for surface water waves
can be described as follows. Suppose a wave field with a prominent
amplitude $a$ and characteristic wavelength $l$ is to be studied. The
KdV equation is known to produce a good approximation of the
evolution of the waves if the amplitude of the waves is small and
the wavelength is large when compared to the undisturbed depth,
and if in addition, the two non-dimensional quantities $a / h_0$ and
$h_0^2 / l^2$ are of similar size. The latter requirement can be written in
terms of the Stokes number as
%
\begin{equation}
\mathcal{S} = \frac{a l^2}{h_0^3} \sim 1. 
\label{stockes-number}
\end{equation}
%
While the KdV equation is a good model for surface waves if $\mathcal{S} \sim 1$, 
one notorious problem with the KdV equation is that it does
not model accurately the dynamics of shorter waves. Recognizing
this shortcoming of the KdV equation, Whitham proposed to use
the same nonlinearity as the KdV equation, but couple it with a
linear term which mimics the linear dispersion relation of the full
water-wave problem. Thus, at least in theory, the Whitham
equation can be expected to yield a description of the dynamics
of shorter waves which is closer to the solutions of the more
fundamental Euler equations which govern the flow.


The Whitham equation has been studied from a number of
vantage points during recent years. In particular, the existence of
traveling and solitary waves has been established in \citep{Ehrnstrom2012, Ehrnstrom2009}.
Well posedness of a similar equation was investigated in \citetext{\citealp[]{Lannes2013}-\citealp[]{Klein2015}}, and a
model with variable depth has been studied numerically in \cite{Minzoni2013}.
Moreover, it has been shown in \citep{Hur2015, Sanford2014} that periodic solutions of
the Whitham equation feature modulational instability for short
enough waves in a similar way as small-amplitude periodic wave
solutions of the water-wave problem. However, even though the
equation is routinely mentioned in texts on nonlinear waves \citep{Drazin89, Whitham74}, it appears that the performance of the Whitham equation in
the description of surface water waves has not been investigated
so far.


The purpose of the present article is to give an asymptotic
derivation of the Whitham equation as a model for surface water
waves, and to confirm Whitham’s expectation that the equation is
a fair model for the description of time-dependent surface water
waves. For the purpose of the derivation, we introduce an exponential scaling regime in which the Whitham equation can be derived asymptotically from an approximate Hamiltonian principle
for surface water waves. In order to motivate the use of this scaling,
note that the KdV equation has the property that wide classes of
initial data decompose into a number of solitary waves and small-
amplitude dispersive residue \cite{Ablowitz81}. For the KdV equations, solitary-
wave solutions are known in closed form, and are given by
%
\begin{equation}
	\eta = \frac{a}{h_0}\mbox{sech}^2\left(\sqrt{\frac{3a}{4h_0^3}(x - ct)}\right)
	\label{kdv-soliton}
\end{equation}
%
for a certain wave celerity $c$. These waves clearly comply with
the amplitude–wavelength relation $a / h_0 \sim h_0^2 / l^2$ which was
mentioned above. It appears that the Whitham equation – as indeed do many other nonlinear dispersive equations – also has
the property that broad classes of initial data rapidly decompose
into ordered trains of solitary waves (see Fig. 1). Quantifying the
amplitude–wavelength relation for these solitary waves yields an
asymptotic regime which is expected to be relevant to the validity
of the Whitham equation as a water wave model.


As the curve fit in the right panel of Fig. 1 shows, the relationship
between wavelength and amplitude of the Whitham solitary
waves can be approximately described by the relation $a/h_0 \sim e^{\kappa(l/h_0)^{\nu}}$ for certain values of $\kappa$ and $\nu$. Since the Whitham solitary
waves are not known in exact form, the values of $\kappa$ and $\nu$ have to be
found numerically. Then one may define a Whitham scaling regime
%
\begin{equation}
	\mathcal{W}(\kappa, \nu) = \frac{a}{h_0}e^{\kappa(l/h_0)^{\nu}} \sim 1,
	\label{whitham-regime}
\end{equation}
%
and it will be shown in Sections 2 and 3 that this scaling can be used
advantageously in the derivation of the Whitham equation. The
derivation proceeds by examining the Hamiltonian formulation of
the water-wave problem due to Zakharov, Craig and Sulem \citep{Craig1993, Zakharov1968},
and by restricting to wave motion which is predominantly in the
direction of increasing values of x. The approach is similar to the
method of \cite{Craig1994}, but relies on the new relation \eqref{whitham-regime}.


First, in Section 2, a Whitham system is derived which allows for
two-way propagation of waves. The Whitham equation is found in
Section 3. Finally, in Section 4, a comparison of modeling properties
of the KdV and Whitham equations is given. The comparison also
includes the regularized long-wave equation
%
\begin{equation}
	\eta_t + c_0 \eta_x + \frac{3}{2} \frac{c_0}{h_0} \eta \eta_x - \frac{1}{6} h_0^2 			\eta_{xxt} = 0,
	\label{regularized-long-wave}
\end{equation}
%
which was put forward in \cite{Peregrine1966} and studied in depth in \cite{Benjamin1972}, and
which is also known as the BBM or PBBM equation. The linearized
dispersion relation of this equation is not an exact match to the
dispersion relation of the full water-wave problem, but it is much
closer than the KdV equation, and it might also be expected that
this equation may be able to model shorter waves more success-
fully than the KdV equation. In order to obtain an even better match
of the linear dispersion relation, one may make use of Pad\'e  expansions. In the context of simplified evolution equations, this
approach was pioneered in \cite{Witting1984}. For uni-directional models, this
approach was advocated in \cite{Fetecau2005}, and in particular, the equation
based on the Pad\'e (2,2) approximation was studied in depth. In dimensional variables, this model takes the form
%
\begin{equation}
	\eta_t + c_0 \eta_x + \frac{3}{2} \frac{c_0}{h_0} \eta \eta_x
	- \frac{3}{20} c_0 h_0^2 \eta_{xxx} - \frac{19}{60} h_0^2 \eta_{xxt} = 0.
	\label{pade-2-2}
\end{equation}
%
The dispersion relations for the KdV, BBM and Pad\'e (2,2) models are respectively
%
\begin{align*}
	c(k) & = c_0 - \frac{1}{6} c_0 h_0^2 k^2 \qquad \mbox{(KdV)}, \\
	c(k) & = c_0 \frac{1}{1 + \frac{1}{6} h_0^2 k^2} \qquad ~\mbox{(BBM)}, \\
	c(k) & = c_0 \frac{1 + \frac{3}{20} h_0^2 k^2}{1 + \frac{19}{60} h_0^2 k^2} \qquad 			\mbox{(Pad\'e (2,2))}.
	\label{dispersive-relations}
\end{align*}
%
These approximate dispersion relations are compared to the full
dispersion relation in Fig. 2. It appears clearly that the Pad\'e (2,2)
approximation remains much closer to the full dispersion relation than the dispersion relations based on either the linear KdV
or linear BBM equations. As will be seen in Section 4, solutions of
both the Whitham and Pad\'e (2,2) equations give closer approximations to solutions of the full Euler equations than either the KdV or
BBM equations in most cases investigated. However, the Whitham
equation still keeps a slight edge over the Pad\'e model.

\subsection{Derivation of evolution systems of Whitham type.}

The surface water-wave problem is generally described by the
Euler equations with slip conditions at the bottom, and kinematic
and dynamic boundary conditions at the free surface. Assuming
weak transverse effects, the unknowns are the surface elevation
$\eta ( x , t )$, the horizontal and vertical fluid velocities $u_1 ( x , z , t )$
and $u_2 ( x , z , t )$, respectively, and the pressure $P_( x , z , t )$. If the
assumption of irrotational flow is made, then a velocity potential
$\phi( x , z , t )$ can be used. In order to nondimensionalize the problem,
the undisturbed depth $h_0$ is taken as a unit of distance, and the
parameter $\sqrt{h_0 / g}$ as a unit of time. For the remainder of this article,
all variables appearing in the water-wave problem are considered
as being non-dimensional. The problem is posed on a domain
$\lbrace( x , z )^T \in R^2 ~|~ 1 < z < \eta( x , t )\rbrace$ which extends to infinity in the
positive and negative $x$-direction. Due to the incompressibility of
the fluid, the potential then satisfies Laplace’s equation in this
domain. The fact that the fluid cannot penetrate the bottom is
expressed by a homogeneous Neumann boundary condition at the
flat bottom. Thus we have
%
\begin{align*}
	\phi_{xx} + \phi_{zz} = 0 ~ &\mbox{in} ~ -1< z < \eta(x,t) \\
	\phi_{zz} = 0 ~ &\mbox{on} ~ z = -1.
\end{align*}
%
The pressure is eliminated with the help of the Bernoulli equation,
and the free-surface boundary conditions are formulated in terms
of the potential and the surface excursion by
%
\begin{equation*}
	\left.
		\begin{array}{rc}
			\eta_t+\phi_x\eta_x-\phi_z
			& =0,
			\\
			\phi_t+\frac{1}{2} \left( \phi^2_x+\phi^2_z \right) + \eta
			& = 0, 
		\end{array}
	\right\}
	\mbox{on} \ z=\eta(x,t).
\end{equation*}
%
The total energy of the system is given by the sum of kinetic energy and potential energy, and normalized such that the potential
energy is zero when no wave motion is present at the surface. Accordingly the Hamiltonian function for this problem is
%
\begin{equation}
	H = \int _{\mathbb{R}} \int_0^\eta z \, dz dx +
	\int _{\mathbb{R}} \int_{-1}^\eta \frac{1}{2} |\nabla \phi|^2 \, dz dx.
\end{equation}
%
Defining the trace of the potential at the free surface as $\Phi( x , t ) = \phi( x , \eta( x , t ), t )$, one may integrate in $z$ in the first integral and use
the divergence theorem on the second integral in order to arrive at
the formulation
%
\begin{equation}
	H  = \frac{1}{2} \int_{\mathbb{R}} \left[ \eta^2 + \Phi G(\eta) \Phi \right] \, dx.
\end{equation}
%
This is the Hamiltonian formulation of the water wave problem as
found in \citep{Craig1993, Petrov1964, Zakharov1968}, and written in terms of the Dirichlet–Neumann
operator $G (\eta)$ . As shown in \cite{Nicholls2001}, the Dirichlet–Neumann operator
can be expanded in a series of the form
%
\[
	G(\eta)\Phi = \sum_{j=0}^\infty G_j(\eta) \Phi
	.
\]
%
In order to proceed, we need to understand the first few terms in
this series. As shown in \citep{Craig1994, Craig1993}, the first two terms in this series
can be written with the help of the operator $D = -i \partial_x$ as
%
\[
	G_0(\eta)= D\tanh(D)
	, \qquad 
	G_1(\eta)= D\eta D - D\tanh(D) \eta D\tanh(D)
	. 
\]
%
Note that it can be shown that the terms $G_j (\eta)$ for $j \geq 2$ are of
quadratic or higher-order in $\eta$, and will therefore not be needed in
the current analysis.


It will be convenient for the present purpose to formulate the
Hamiltonian in terms of the dependent variable $u = \Phi_x$. To this
end, we define the operator $\mathcal{K} (\eta)$ by
%
\begin{equation*}
	G(\eta) = D \mathcal{K}(\eta) D
	.
\end{equation*}
%
As was the case with $G (\eta)$, the operator $\mathcal{K}(\eta)$ can be expanded in
a Taylor series around zero as
%
\[
	\mathcal{K}(\eta) = \sum_{j=0}^\infty \mathcal{K}_j(\eta)
	, \qquad
	\mathcal{K}_j(\eta) = D^{-1}G_j(\eta)D^{-1}
	.
\]
%
In particular, note that $\mathcal{K}_0 = \frac{\tanh D}{D}$. In non-dimensional variables, we write the operator with the integral kernel $K_{h_0}$ as $K = \sqrt{\frac{\tanh D}{D}}$, so that we have $\mathcal{K}_0 = K^2$. The Hamiltonian is then expressed as 
%
\begin{equation}
	H  = \frac{1}{2} \int_{\mathbb{R}} \left[ \eta^2 + u \mathcal{K}(\eta) u \right] \, dx.
	\label{scaled-Hamiltonian}
\end{equation}
%
The water-wave problem can then be written as a Hamiltonian
system using the variational derivatives of $H$ and posing the
Hamiltonian equations
%
\begin{equation}
	\eta_t = - \partial_x \frac{\delta H}{\delta u}, \qquad u_t = -\partial_x \frac{\delta H}{\delta \eta}.
\end{equation}
%
This system is not in canonical form as the associated structure
map $J_{\eta, u}$ is symmetric:
%
\begin{equation}
	J_{\eta, u} =
	\begin{pmatrix}
		0 & -\partial_x
		\\
		-\partial_x & 0
	\end{pmatrix}.
\end{equation}
%
We now proceed to derive a system of equations which is similar to
the Whitham equation \eqref{whitham-equation}, but allows bi-directional wave propagation. This system will be a stepping stone on the way to a derivation
of \eqref{whitham-equation}, but may also be of independent interest. Consider a wave-
field having a characteristic wavelength $l$ and a characteristic amplitude $a$. Taking into account the nondimensionalization, the two
scalar parameters $\lambda = l / h_0$ and $ \alpha = a / h_0$ appear. In order to introduce the long-wave and small amplitude approximation into the
non-dimensional problem, we use the scaling $\tilde{x}= \frac{1}{\lambda} x$, and $ \eta = \alpha \tilde{\eta}$.
This induces the transformation $\tilde{D} = \lambda D = -\lambda i \partial_x $. If the energy
is nondimensionalized in accord with the nondimensionalization
mentioned earlier, then the natural scaling for the Hamiltonian is
$ \tilde{H}= \alpha^2 H$. In addition, the unknown $u$ is scaled as $u = \alpha \tilde{u}$. The
scaled Hamiltonian \eqref{scaled-Hamiltonian} is then written as
%
\begin{multline*}
	\tilde{H} = \frac{1}{2} \int_{\R} \teta^2 \, dx
 + \frac{1}{2} \int_{\R} \tilde{u} \left[ 1 - \Sfrac{1}{3} \lambda^{-2} \tilde{D}^2 + \cdots \right] \tilde{u} \, dx  
 + \frac{\alpha}{2} \int_{\R}  \teta  \tilde{u}^2 \, dx
\\
 - \frac{\alpha}{2} \int_{\R}  \tilde{u} \left[ \lambda^{-1} \tilde{D} - \Sfrac{1}{3} \lambda^{-3}\tilde{D}^3 +  \cdots \right] \teta 
   \left[ \lambda^{-1} \tilde{D} - \Sfrac{1}{3} \lambda^{-3}\tilde{D}^3 + \cdots \right] \tilde{u} \, dx.
\end{multline*}
%
Let us now introduce the small parameter $\mu= \frac{1}{\lambda}$,
and assume for simplicity that $\alpha = e^{- \kappa / \mu^\nu}$, which corresponds to the
case where $\mathcal{W}(\kappa,\nu) = 1$. 
Then the Hamiltonian can be written in the following form:
%
\begin{multline*}
\tH = \ohf \int_{\R} \teta^2 \, dx
        + \ohf \int_{\R} \tu \left[ 1 - \Sfrac{1}{3} \mu^{2} \tD^2 + \cdots \right] \tu \, dx  
+ \frac{e^{-\kappa/\mu^\nu}}{2} \int_{\R}  \teta  \tu^2 \, dx
\\
- \frac{e^{-\kappa/\mu^\nu}}{2} \int_{\R}  \tu \left[ \mu \tD - \Sfrac{1}{3} \mu^{3}\tD^3 +  \dots\right]
                      \teta \left[ \mu \tD - \Sfrac{1}{3} \mu^{3}\tD^3 + \cdots \right] \tu \, dx.
\end{multline*}
%
Disregarding terms of order $\scrO(\mu^2 e^{-\kappa/\mu^\nu})$, but not of order
$\scrO(e^{\sca})$ yields the expansion
\begin{equation}
\tH = \ohf \int_{\R} \teta^2 \, dx 
    + \ohf \int_{\R} \tu \left[ 1 - \Sfrac{1}{3} \mu^{2} \tD^2 + \dots \right] \tu \, dx  
    + \frac{e^{\sca}}{2} \int_{\R}  \teta  \tu^2 \, dx.   
\end{equation}
%
Note that by taking $\mu$ small enough, an arbitrary number of terms of algebraic order 
in $\mu$ may be kept in the asymptotic series, 
so that the truncated version of the Hamiltonian in dimensional 
variables may be written as
%
\begin{equation}\label{Hamiltonian-eta-u-truncated}
	H   = \ohf \int_\R \left[ \eta^2 + u \mathcal{K}_0^N (\eta) u + u \eta u \right] \, dx dz.
\end{equation}
%
However, the difference between $\mathcal{K}_0$ and $\mathcal{K}_0^N$
is below the order of approximation, so that it is possible
to formally define the truncated Hamiltonian with
$\mathcal{K}_0$ instead of $\mathcal{K}_0^N$.
%
Hence, the Whitham system is obtained from the 
Hamiltonian \eqref{Hamiltonian-eta-u-truncated} as follows:
%
\begin{align}
	\label{sys1}
	\eta_t = - \partial_x \frac{\delta H}{\delta u} 
       &= -  \mathcal{K}_0 u_x - (\eta u)_x, \\
	\label{sys2}
	u_t = -\partial_x \frac{\delta H}{\delta \eta}
       &= - \eta_x - u u_x.
\end{align}
%
One may also derive a higher-order equation by keeping terms of order 
$\scrO(\mu^2 e^{\sca})$, but discarding terms of order $\scrO(\mu^4 e^{\sca})$.
In this case we find the system
%
\begin{align*}
	\eta_t & = -  \mathcal{K}_0 u_x - (\eta u)_x - (\eta u_x)_{xx}, \\
	u_t    & = - \eta_x - u u_x + u_x u_{xx}.
\end{align*}
%
% *************************************************************************************
\subsection{Derivation of evolution equations of Whitham type.}
% *************************************************************************************
%
In order to derive the Whitham equation for uni-directional wave propagation,
it is important to understand how solutions of the Whitham system \eqref{sys1}-\eqref{sys2}
can be restricted to either left or right-going waves.
As it will turn out, if  $\eta$ and $u$ are such that $\eta = K u$, 
then this pair of functions represents a solution
of \eqref{sys1}-\eqref{sys2} which is propagating to the right. 
%
% *************************************************************************************
Indeed, let us analyze the relation between $\eta$ and $u$ in the linearized Whitham system
\begin{align}
\label{linsys1}
\eta_t & = -  \mathcal{K}_0 u_x, \\
\label{linsys2}
u_t    & = - \eta_x.
\end{align}
%
Considering a solution of the system \eqref{linsys1}-\eqref{linsys2} in the form
\begin{equation}
\eta(x,t) = A e^{(i\xi x - i \omega t)}, \qquad u(x,t) = B e^{(i\xi x - i \omega t)}.
\end{equation}
%
\noindent 
gives rise to the matrix equation
\begin{equation}
\begin{pmatrix}
-\omega & \frac{\tanh{\xi}}{\xi} \xi \\
\xi  & -\omega
\end{pmatrix}
\begin{pmatrix}
A \\
B
\end{pmatrix} =
\begin{pmatrix}
0 \\
0
\end{pmatrix}.
\label{matrix-equation}
\end{equation}
%
If existence of a nontrivial solution of this system is to be guaranteed,
the determinant of the matrix has to be zero, so that we have
$\omega^2 - \frac{\tanh{\xi}}{\xi} \xi^2 = 0$. Defining the phase speed
as $c = \omega / \xi$, we obtain the dispersion relation
\begin{equation}
c = \pm \sqrt{{\textstyle \frac{\tanh{\xi}}{\xi}}}.
\end{equation}
%
%
The choice of $c > 0$ corresponds to right-going wave solutions of the system 
\eqref{linsys1}-\eqref{linsys2},
and the relation between $\eta$ and $u$ can be deduced from \eqref{linsys2}.
%
%
Accordingly, it is expedient to separate solutions
into a right-going part $r$ and a left-going part $s$
which are defined by
\begin{align*}
r = \ohf (\eta + K  u), \qquad
s = \ohf (\eta - K  u). 
\end{align*}
According to the transformation theory detailed in \cite{Craig2005}, 
if the unknowns $r$ and $s$ are used instead of $\eta$ and $u$,
the structure map changes to 
\begin{equation}
J_{r,s} = \left( \frac{\partial F}{\partial (\eta, u)} \right) J_{\eta, u}  
\left( \frac{\partial F}{\partial (\eta, u)} \right)^T
= 
\begin{pmatrix}
-\ohf \partial_x K & 0\\
0 & \ohf \partial_x K
\end{pmatrix}.
\end{equation}
%
\noindent 
We now use the same scaling for both dependent and independent variables as before.
Thus we have $r = \alpha\tr$ and $s = \alpha\ts$. 
%
%
%
% *************************************************************************************
%
The Hamiltonian is written in terms of $\tr$ and $\ts$ as
\begin{multline*}
\tH = \ohf \int_{\R} (\tr+\ts)^2 \, dx
\\
 + \ohf \int_{\R} \tK^{-1}(\tr-\ts) \left[ 1 - \Sfrac{1}{3} \mu^2 \tD^2 + \cdots \right] \tK^{-1}(\tr-\ts) \, dx
%\\
 + \frac{\alpha}{2} \int_{\R}  (\tr+\ts) \left(\tK^{-1}(\tr-\ts)\right)^2 \, dx
\\
 - \frac{\alpha}{2} \int_{\R}  \tK^{-1}(\tr-\ts) \left[ \mu \tD - \Sfrac{1}{3} \mu^3\tD^3 +  \cdots \right] (\tr+\ts)
   \left[ \mu \tD - \Sfrac{1}{3} \mu^3\tD^3 + \cdots \right] \tK^{-1}(\tr-\ts) \, dx. 
\end{multline*}
%
%
%
% *************************************************************************************
%
%
Following the transformation rules, the structure map transforms to
$J_{\tr,\ts} = 1/\ta^2 J_{r,s}$. 
In addition, the time scaling $t = \lambda \ttt$ is employed. 
Since the focus is on right-going solutions, the equation to be considered is
\begin{equation}
\lambda \tr_{\ttt} 
= -\frac{1}{2\ta^2}\lambda\partial_{\tx} \tK \left[ \frac{\delta \big( \ta^2 \tH \big)}{\delta \tr} \right].
\end{equation}
So far, this equation is exact. If we now assume that 
%the waves are predominantly right-going, and 
$s$ is of the order of
$\scrO(\mu^2 e^{\sca})$, then the equation for $\tr$ is
\begin{multline*}
\tr_{\ttt} = -\frac{1}{2}\partial_{\tx} \big[ 1 - \Sfrac{1}{6} \mu^2 \tD^2 + \cdots \big] 
           \bigg\{ 2 \tr + \frac{\alpha}{2} \left( \big[1 + \Sfrac{1}{6} \mu^2 \tD^2 + \cdots \big] 
           \tr \right)^2 
%\right.
\\
+ \alpha \big[ 1 + \Sfrac{1}{6} \mu^2 \tD^2 + \cdots \big] 
   \left( \tr \, \big[1 + \Sfrac{1}{6} \mu^2 \tD^2 + \cdots \big]  \tr  \right)  
- \frac{\alpha}{2} \left( \big[ \mu \tD - \Sfrac{1}{3} \mu^3 \tD^3 + \cdots \big] 
    \big[ 1 + \Sfrac{1}{6} \mu^2 \tD^2 + \cdots \big]  \tr  \right)^2
\\
%\left. 
- \alpha \big[ \mu \tD - \Sfrac{1}{3} \mu^3 \tD^3 + \cdots \big] \big[1 + \Sfrac{1}{6} \mu^2 \tD^2 + \cdots \big] 
 \left( \tr \big[ \mu \tD - \Sfrac{1}{3} \mu^3 \tD^3 + \cdots \big] [1 + \Sfrac{1}{6} \mu^2 \tD^2 
    + \cdots \big] \tr \right) \bigg\} + \scrO(\alpha \mu^2).
\end{multline*}
As in the case of the Whitham system, we use $\ta = \scrO(e^{- \kappa / \mu^\nu})$,
and disregard terms of order $\scrO(\mu^2 e^{\sca})$, but not of order $\scrO(e^{\sca})$.
This procedure yields the Whitham equation \eqref{whitham-equation} 
which is written in nondimensional variables as
\begin{equation*}
r_t = -K r_x - \frac{3}{2} r r_x.
\end{equation*}
As was the case for the system found in the previous section, it is also 
possible here to include terms of order $\scrO(\mu^2 e^{-\kappa/\mu^\nu})$, resulting
in the higher-order equation
\begin{equation*}
r_t = -K r_x - \frac{3}{2} r r_x - \frac{13}{12} r_x r_{xx} - \frac{5}{12} r r_{xxx}.
\end{equation*}
%
%
% *************************************************************************************
\subsection{Numerical results.}
% *************************************************************************************
%
In this section, the performance of the Whitham
equation as a model for surface water waves 
is compared to the KdV equation \eqref{kdv-equation}, the BBM equation \eqref{regularized-long-wave},
and the Pad\'e (2,2) equation \eqref{pade-2-2}.
For this purpose initial data are imposed, the Whitham, KdV, BBM, and Pad\'e equations are solved numerically,
and the solutions are compared to numerical solutions of the full Euler equations
with free-surface boundary conditions. We continue to work in normalized
variables, such as stated in the beginning of Section 2.

The numerical treatment of the three model equations is 
by a standard pseudo-spectral scheme,
such as explained in \cite{Ehrnstrom2013,Fornberg1978} for example. For the time stepping,
an efficient fourth-order implicit method developed in \cite{DeFrutos1992} is used. 
The numerical treatment of the free-surface problem for the Euler equations
is based on a conformal mapping of the fluid domain into a rectangle.
In the time-dependent case, this method has roots in the work of Ovsyannikov \cite{Ovsyannikov1974}, 
and was later used in \cite{Dyachenko1996} and \cite{Li2004}. 
The particular method used for the numerical experiments reported here 
is a pseudo-spectral scheme which is detailed in \cite{Mitsotakis2014}.

Initial conditions for the Euler equations are chosen in such a way that 
the solutions are expected to be right moving. This is achieved by
posing an initial surface disturbance $\eta_0(x)$ together with the
trace of the potential $\Phi(x) = \int_0^x \eta_0(x')\, dx'.$
In order to normalize the data, we choose $\eta_0(x)$ in such a way that
the average of $\eta_0(x)$ over the computational domain is zero.
The experiments are performed %with two different initial functions and 
with several different amplitudes $\alpha$ and wavelengths $\lambda$
(for the purpose of this section, we define the 
wavelength $\lambda$ as the distance between 
the two points $x_1$ and $x_2$ at which $\eta_0(x_1) = \eta_0(x_2) = \alpha/2$).
Both positive and negative initial disturbances are considered.
While disturbances with positive main part have been studied widely, 
an initial wave of depression is somewhat more exotic, but nevertheless important,
as shown for instance in \cite{Hammack1974}.
A summary of the experiments' settings is given in Table 1.
Experiments run with an initial wave of elevation are
labeled as {\it positive},
and experiments run with an initial wave of depression are
labeled as {\it negative}.
The domain for the computations is $-L\leq x \leq L$, with $L = 50$. 
The function initial data in the {\it positive} cases is given by
\begin{equation}
\eta_0(x) =  \alpha \sech^2(f(\lambda)x)-C, \label{test_1}
\end{equation}
where 
\begin{equation*}
f(\lambda) = \frac{2}{\lambda}\log\left({\textstyle \frac{1+\sqrt{1/2}}{\sqrt{1/2}}}\right), 
%\end{equation}
\ \mbox{ and } \
%\begin{equation}
%\label{C_const},
C = \frac{1}{L}\frac{\alpha}{f(\lambda)}\tanh \left({\textstyle \frac{L}{f(\lambda)}}\right).
\end{equation*}
and $C$ and $f(\lambda)$ are chosen so that 
$\int_{-L}^{L}\eta_0(x)dx = 0$, and
the wavelength $\lambda$ is the distance between the 
two points $x_1$ and $x_2$ at which $\eta_0(x_1) = \eta_0(x_2) = a/2$.
The velocity potential in this case is given by
\begin{equation}
\label{velocity1}
\Phi(x) = \frac{\alpha}{f(\lambda)}\tanh(f(\lambda)x)-Cx. 
\end{equation}
In the {\it negative} case, the initial data are given by
\begin{equation*}
%\label{test_2}
\eta_0(x) = - \alpha \sech^2(f(\lambda)x) + C. 
\end{equation*}
The definitions for $f(\lambda)$ and $C$ are the same, and the velocity potential is
\begin{equation*}
%\label{velocity2}
\Phi(x) = -\frac{\alpha}{f(\lambda)}\tanh(f(\lambda)x)+Cx. 
\end{equation*}
%
\begin{table}
\begin{center}
	\begin{tabular}{| c | c | c | c |}
	\hline
	Experiment & Stokes number & $\alpha$ & $\lambda$ \\
	\hline
	A	& 0.2  &  	 0.1 &	$\sqrt{2}$ 	\\	
	B	& 0.2  & 	 0.2 &	1 			\\
	C	& 1 	   & 	 0.1 &	$\sqrt{10}$	\\
	D	& 1    & 	 0.2 &	$\sqrt{5}$	\\	
	E	& 5    & 	 0.1 &	$\sqrt{50}$ 	\\
	F	& 5    & 	 0.2 &	5			\\
	\hline
	\end{tabular}
\end{center}
\caption{Summary of the Stokes number, nondimensional amplitude and
nondimensional wavelength of the initial data used in the numerical
experiments.}
\end{table}
%
In figures Figs. 3 and 4, the time evolution of a wave with an initial narrow peak
and one with an initial narrow depression at the center is shown. 
%
The amplitude is $\alpha=0.2$, and the wavelength is $\lambda = \sqrt{5}$. 
In Fig. 3 , the time evolution
according to the Euler, Whitham, KdV and BBM equations are shown,
and in Fig. 4, the time evolution
according to the Euler, Whitham, and Pad\'e (2,2) equations are shown.

It appears in Fig. 3 
that the KdV equation produces a significant number of spurious oscillations,
the BBM equation also produces a fair number of spurious oscillations, and
the Whitham equation produces fewer small oscillations than Euler equations.
Moreover, while the highest peak in the upper panels in Fig. 3 is
underpredicted by the KdV and BBM equation, the Whitham equation
produces peaks which are slightly too high.
In the case of an initial depression, the Whitham equation also
produces some peaks which are too high, but on the other hand,
the KdV and the BBM equations introduce phase errors in the
main oscillations. 
As is visible in Fig. 4,
the Pad\'e (2,2) equation reproduces the leading wave fairly accurately,
but overpredicts the trailing oscillations in the case of a positive disturbance,
and underpredicts the trailing oscillations in the case of a negative initial disturbance.
Nevertheless, since the Pad\'e (2,2) does not introduce a phase error, the overall performance
is better than that of the KdV and BBM equations.

In order to compare the performance of the four approximate equations
in a more quantitative manner, the discrepancies between solutions of the model equations
and the prediction due to solving the Euler equations are measured in an integral norm.
In the center right panels of Figs. 5 and 6, 
the computations from Figs. 3 and 4 are
summarized by plotting the normalized $L^2$-error between the KdV, BBM, Pad\'e and Whitham,
respectively, and the Euler solutions as a function of non-dimensional time.
Using this quantitative measure of comparison, it appears that the Whitham
equation gives the best overall rendition of the free surface dynamics
predicted by the Euler equations.

In the center left panels of Figs. 5 and 6, 
a similar computation with $\mathcal{S} = 1$, but smaller amplitude is analyzed. 
Also in these cases, it appears that the Whitham
equation gives a good approximation to the corresponding Euler solutions, and in
particular, a much better approximation than either the KdV or the BBM equation.
The Pad\'e equation also does better than both KdV and BBM equations, but not better
than the Whitham equation.

Figs. 5 and 6 show comparisons in several other cases of both positive
and negative initial amplitude, 
and Stokes numbers $\mathcal{S} = 0.2$, $\mathcal{S} = 1$ and $\mathcal{S} = 5$.
In most cases, the normalized $L^2$-error between the Whitham and Euler solutions is similar
or smaller than the errors between the Euler solutions and the other three model equations.
However, the Pad\'e equation generally outperforms both the KdV and the BBM equation by some measure.

The only case in this study in which the KdV, BBM and Pad\'e equations outperform
the Whitham equation is in the case of very long positive initial disturbances.
The case when $\mathcal{S} = 5$ is shown in the lower panels of Fig.5. 
However, even in this case, the Whitham equation yields approximations of the
Euler solutions which are similar or better than in the case of smaller wavelengths.
In addition, in the case of negative initial data, the performance
of the Whitham equation is on par with the KdV, BBM and Pad\'e equations 
in the case when $\mathcal{S} = 5$ (lower panels of Fig.5).

%
%
% *************************************************************************************
\subsection{Conclusion.}
% *************************************************************************************
%
In this article, the Whitham equation \eqref{whitham-equation}
has been studied as an approximate model equation for wave motion
at the surface of a perfect fluid. Numerical integration of the
equation suggests that broad classes of initial data decompose
into individual solitary waves. The wavelength-amplitude ratio
of these approximate solitary waves has been studied, and it 
was found that this ratio can be described by an exponential
relation of the form
$\frac{a}{h_0} \sim e^{-\kappa(l/h_0)^\nu}$.
Using this scaling in the Hamiltonian formulation of the water-wave problem,
a system of evolution equations has been derived which contains the exact
dispersion relation of the water-wave problem in its linear part.
Restricting to one-way propagation, the Whitham equation emerged
as a model which combines the usual quadratic nonlinearity 
with one branch of the exact dispersion relation of the water-wave problem.
The performance of the Whitham equation in the approximation of
solutions of the Euler equations free-surface boundary conditions
was analyzed, and compared to the performance of the KdV and BBM equations,
and to the Pad\'e (2,2) model.
It was found that the Whitham equation gives a more faithful representation 
of the solutions of the Euler equations than either the KdV or the BBM equations, 
except in the case of very long waves with positive main part. In this last case,
the KdV and BBM equations have the upper hand over the Whitham equation.
The Pad\'e (2,2) model also outperforms the KdV and BBM equations,
but does not quite as well as the Whitham equation for shorter waves and negative
disturbances. However, in the case of very long waves with positive main part,
the Pad\'e model stays on par with the the KdV and BBM equations.
