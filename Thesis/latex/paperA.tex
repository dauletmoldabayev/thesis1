%%%%%%%%%%%%%%%%%%%%%%%%%%%%%%%%%%%%%%%%%%%%%%%%%%%%%%%%%%%%%%%%%%%%%%%%%%%%%%%%%%%%%%%%%%%
\section{The Whitham equation as a model for surface water waves}
%%%%%%%%%%%%%%%%%%%%%%%%%%%%%%%%%%%%%%%%%%%%%%%%%%%%%%%%%%%%%%%%%%%%%%%%%%%%%%%%%%%%%%%%%%%
\subsection{Introduction.}
The Whitham equation was proposed as an alternate model equation for the simplified description of unidirectional 
wave motion at the surface of an inviscid fluid. As the Whitham equation incorporates the full
linear dispersion relation of the water wave problem, it is thought to provide a more faithful description
of shorter waves of small amplitude than traditional long wave models such as the KdV equation.
In this work, we identify a scaling regime in which the Whitham equation can be derived from the
Hamiltonian theory of surface water waves. A Hamiltonian system of Whitham type allowing for two-
way wave propagation is also derived. The Whitham equation is integrated numerically, and it is shown
that the equation gives a close approximation of inviscid free surface dynamics as described by the
Euler equations. The performance of the Whitham equation as a model for free surface dynamics is also
compared to different free surface models: the KdV equation, the BBM equation, and the Pad\'e (2,2) model.
It is found that in a wide parameter range of amplitudes and wavelengths, the Whitham equation performs
on par with or better than the three considered models.


In its simplest form, the water-wave problem concerns the
flow of an incompressible inviscid fluid with a free surface over
a horizontal impenetrable bed. In this situation, the fluid flow
is described by the Euler equations with appropriate boundary
conditions, and the dynamics of the free surface are of particular
interest in the solution of this problem.
There are a number of model equations which allow the
approximate description of the evolution of the free surface
without having to provide a complete solution of the fluid flow
below the surface. In the present contribution, interest is focused

on the derivation and evaluation of a nonlocal water-wave model
known as the Whitham equation. The equation is written as

\begin{equation}
\eta_t + \frac{3}{2} \frac{c_0}{h_0} \eta \eta_x + K_{h_0} \ast \eta_x = 0, 
\label{whitham-equation}
\end{equation}
where the convolution kernel $K_{h_0}$ is given in terms of the Fourier
transform by

\begin{equation}
\mathcal{F}K_{h_0} (\xi) =  \sqrt{\frac{g \tanh(h_0\xi)}{\xi}}.
\label{whitham-kernel}
\end{equation}
$g$ is the gravitational acceleration, $h_0$ is the undisturbed depth of
the fluid, and $c_0 = \sqrt{gh_0}$ is the corresponding long-wave speed.
The convolution can be thought of as a Fourier multiplier operator,
and \eqref{whitham-kernel} represents the Fourier symbol of the operator.
The Whitham equation was proposed by Whitham \cite{Whitham1967} as an
alternative to the well known Korteweg–de Vries (KdV) equation

\begin{equation}
\eta_t + c_0 \eta_x + \frac{3}{2} \frac{c_0}{h_0} \eta \eta_x + \frac{1}{6} c_0 h_0^2 \eta_{xxx} = 0.
\label{kdv-equation}
\end{equation}


The validity of the KdV equation as a model for surface water waves
can be described as follows. Suppose a wave field with a prominent
amplitude $a$ and characteristic wavelength $l$ is to be studied. The
KdV equation is known to produce a good approximation of the
evolution of the waves if the amplitude of the waves is small and
the wavelength is large when compared to the undisturbed depth,
and if in addition, the two non-dimensional quantities $a / h_0$ and
$h_0^2 / l^2$ are of similar size. The latter requirement can be written in
terms of the Stokes number as

\begin{equation}
\mathcal{S} = \frac{a l^2}{h_0^3} \sim 1. 
\label{stockes-number}
\end{equation}
While the KdV equation is a good model for surface waves if $\mathcal{S} \sim 1$, 
one notorious problem with the KdV equation is that it does
not model accurately the dynamics of shorter waves. Recognizing
this shortcoming of the KdV equation, Whitham proposed to use
the same nonlinearity as the KdV equation, but couple it with a
linear term which mimics the linear dispersion relation of the full
water-wave problem. Thus, at least in theory, the Whitham
equation can be expected to yield a description of the dynamics
of shorter waves which is closer to the solutions of the more
fundamental Euler equations which govern the flow.

The Whitham equation has been studied from a number of
vantage points during recent years. In particular, the existence of
traveling and solitary waves has been established in \citep{Ehrnstrom2012, Ehrnstrom2009}.
Well
posedness of a similar equation was investigated in \citetext{\citealp[]{Lannes2013}-\citealp[]{Klein2015}}, and a
model with variable depth has been studied numerically in \cite{Minzoni2013}.
Moreover, it has been shown in \citep{Hur2015, Sanford2014} that periodic solutions of
the Whitham equation feature modulational instability for short
enough waves in a similar way as small-amplitude periodic wave
solutions of the water-wave problem. However, even though the
equation is routinely mentioned in texts on nonlinear waves \citep{Drazin89, Whitham74}, it appears that the performance of the Whitham equation in
the description of surface water waves has not been investigated
so far.
The purpose of the present article is to give an asymptotic
derivation of the Whitham equation as a model for surface water
waves, and to confirm Whitham’s expectation that the equation is
a fair model for the description of time-dependent surface water
waves. For the purpose of the derivation, we introduce an expo-
nential scaling regime in which the Whitham equation can be de-
rived asymptotically from an approximate Hamiltonian principle
for surface water waves. In order to motivate the use of this scaling,
note that the KdV equation has the property that wide classes of
initial data decompose into a number of solitary waves and small-
amplitude dispersive residue \cite{Ablowitz81}. For the KdV equations, solitary-
wave solutions are known in closed form, and are given by
\begin{equation}
\eta = \frac{a}{h_0}\mbox{sech}^2\left(\sqrt{\frac{3a}{4h_0^3}(x - ct)}\right)
\label{kdv-soliton}
\end{equation}
for a certain wave celerity $c$. These waves clearly comply with
the amplitude–wavelength relation $a / h_0 \sim h_0^2 / l^2$ which was
mentioned above. It appears that the Whitham equation – as indeed do many other nonlinear dispersive equations – also has
the property that broad classes of initial data rapidly decompose
into ordered trains of solitary waves (see Fig. 1). Quantifying the
amplitude–wavelength relation for these solitary waves yields an
asymptotic regime which is expected to be relevant to the validity
of the Whitham equation as a water wave model.

As the curve fit in the right panel of Fig. 1 shows, the relationship
between wavelength and amplitude of the Whitham solitary
waves can be approximately described by the relation $a/h_0 \sim e^{\kappa(l/h_0)^{\nu}}$ for certain values of $\kappa$ and $\nu$. Since the Whitham solitary
waves are not known in exact form, the values of $\kappa$ and $\nu$ have to be
found numerically. Then one may define a Whitham scaling regime
\begin{equation}
\mathcal{W}(\kappa, \nu) = \frac{a}{h_0}e^{\kappa(l/h_0)^{\nu}} \sim 1,
\label{whitham-regime}
\end{equation}
and it will be shown in Sections 2 and 3 that this scaling can be used
advantageously in the derivation of the Whitham equation. The
derivation proceeds by examining the Hamiltonian formulation of
the water-wave problem due to Zakharov, Craig and Sulem \citep{Craig1993, Zakharov1968},
and by restricting to wave motion which is predominantly in the
direction of increasing values of x. The approach is similar to the
method of \cite{Craig1994}, but relies on the new relation (5).
First, in Section 2, a Whitham system is derived which allows for
two-way propagation of waves. The Whitham equation is found in
Section 3. Finally, in Section 4, a comparison of modeling properties
of the KdV and Whitham equations is given. The comparison also
includes the regularized long-wave equation
\begin{equation}
\eta_t + c_0 \eta_x + \frac{3}{2} \frac{c_0}{h_0} \eta \eta_x - \frac{1}{6} h_0^2 \eta_{xxt} = 0,
\label{regularized-long-wave}
\end{equation}
which was put forward in \cite{Peregrine1966} and studied in depth in \cite{Benjamin1972}, and
which is also known as the BBM or PBBM equation. The linearized
dispersion relation of this equation is not an exact match to the
dispersion relation of the full water-wave problem, but it is much
closer than the KdV equation, and it might also be expected that
this equation may be able to model shorter waves more success-
fully than the KdV equation. In order to obtain an even better match
of the linear dispersion relation, one may make use of Pad\'e  expansions. In the context of simplified evolution equations, this
approach was pioneered in \cite{Witting1984}. For uni-directional models, this
approach was advocated in \cite{Fetecau2005}, and in particular, the equation
based on the Pad\'e (2,2) approximation was studied in depth. In dimensional variables, this model takes the form
\begin{equation}
\eta_t + c_0 \eta_x + \frac{3}{2} \frac{c_0}{h_0} \eta \eta_x
- \frac{3}{20} c_0 h_0^2 \eta_{xxx} - \frac{19}{60} h_0^2 \eta_{xxt} = 0.
\label{pade-2-2}
\end{equation}
The dispersion relations for the KdV, BBM and Pad\'e (2,2) models are respectively
\begin{align*}
c(k) & = c_0 - \frac{1}{6} c_0 h_0^2 k^2 \qquad \mbox{(KdV)}, \\
c(k) & = c_0 \frac{1}{1 + \frac{1}{6} h_0^2 k^2} \qquad ~\mbox{(BBM)}, \\
c(k) & = c_0 \frac{1 + \frac{3}{20} h_0^2 k^2}{1 + \frac{19}{60} h_0^2 k^2} \qquad \mbox{(Pad\'e (2,2))}. \\
\label{dispersive-relations}
\end{align*}

These approximate dispersion relations are compared to the full
dispersion relation in Fig. 2. It appears clearly that the Pad\'e (2,2)
approximation remains much closer to the full dispersion relation than the dispersion relations based on either the linear KdV
or linear BBM equations. As will be seen in Section 4, solutions of
both the Whitham and Pad\'e (2,2) equations give closer approximations to solutions of the full Euler equations than either the KdV or
BBM equations in most cases investigated. However, the Whitham
equation still keeps a slight edge over the Pad\'e model.

\subsection{Derivation of evolution systems of Whitham type.}

The surface water-wave problem is generally described by the
Euler equations with slip conditions at the bottom, and kinematic
and dynamic boundary conditions at the free surface. Assuming
weak transverse effects, the unknowns are the surface elevation
$\eta ( x , t )$, the horizontal and vertical fluid velocities $u_1 ( x , z , t )$
and $u_2 ( x , z , t )$, respectively, and the pressure $P_( x , z , t )$. If the
assumption of irrotational flow is made, then a velocity potential
$\phi( x , z , t )$ can be used. In order to nondimensionalize the problem,
the undisturbed depth $h_0$ is taken as a unit of distance, and the
parameter $\sqrt{h_0 / g}$ as a unit of time. For the remainder of this article,
all variables appearing in the water-wave problem are considered
as being non-dimensional. The problem is posed on a domain
$\lbrace( x , z )^T \in R^2 ~|~ 1 < z < \eta( x , t )\rbrace$ which extends to infinity in the
positive and negative $x$-direction. Due to the incompressibility of
the fluid, the potential then satisfies Laplace’s equation in this
domain. The fact that the fluid cannot penetrate the bottom is
expressed by a homogeneous Neumann boundary condition at the
flat bottom. Thus we have
%
\begin{align*}
	\phi_{xx} + \phi_{zz} = 0 ~ &\mbox{in} ~ -1< z < \eta(x,t) \\
	\phi_{zz} = 0 ~ &\mbox{on} ~ z = -1.
\end{align*}
%
The pressure is eliminated with the help of the Bernoulli equation,
and the free-surface boundary conditions are formulated in terms
of the potential and the surface excursion by
%
\begin{equation*}
	\left.
		\begin{array}{rc}
			\eta_t+\phi_x\eta_x-\phi_z
			& =0,
			\\
			\phi_t+\frac{1}{2} \left( \phi^2_x+\phi^2_z \right) + \eta
			& = 0, 
		\end{array}
	\right\}
	\mbox{on} \ z=\eta(x,t).
\end{equation*}
%
The total energy of the system is given by the sum of kinetic energy and potential energy, and normalized such that the potential
energy is zero when no wave motion is present at the surface. Accordingly the Hamiltonian function for this problem is
%
\begin{equation}
	H = \int _{\mathbb{R}} \int_0^\eta z \, dz dx +
	\int _{\mathbb{R}} \int_{-1}^\eta \frac{1}{2} |\nabla \phi|^2 \, dz dx.
\end{equation}
%
Defining the trace of the potential at the free surface as $\Phi( x , t ) = \phi( x , \eta( x , t ), t )$, one may integrate in $z$ in the first integral and use
the divergence theorem on the second integral in order to arrive at
the formulation
%
\begin{equation}
	H  = \frac{1}{2} \int_{\mathbb{R}} \left[ \eta^2 + \Phi G(\eta) \Phi \right] \, dx.
\end{equation}
%
This is the Hamiltonian formulation of the water wave problem as
found in \citep{Craig1993, Petrov1964, Zakharov1968}, and written in terms of the Dirichlet–Neumann
operator $G (\eta)$ . As shown in \cite{Nicholls2001}, the Dirichlet–Neumann operator
can be expanded in a series of the form
%
\[
	G(\eta)\Phi = \sum_{j=0}^\infty G_j(\eta) \Phi
	.
\]
%
In order to proceed, we need to understand the first few terms in
this series. As shown in \citep{Craig1994, Craig1993}, the first two terms in this series
can be written with the help of the operator $D = -i \partial_x$ as
%
\[
	G_0(\eta)= D\tanh(D)
	, \qquad 
	G_1(\eta)= D\eta D - D\tanh(D) \eta D\tanh(D)
	. 
\]
%
Note that it can be shown that the terms $G_j (\eta)$ for $j \geq 2$ are of
quadratic or higher-order in $\eta$, and will therefore not be needed in
the current analysis.

It will be convenient for the present purpose to formulate the
Hamiltonian in terms of the dependent variable $u = \Phi_x$. To this
end, we define the operator $\mathcal{K} (\eta)$ by
%
\begin{equation*}
	G(\eta) = D \mathcal{K}(\eta) D
	.
\end{equation*}
%
As was the case with $G (\eta)$, the operator $\mathcal{K}(\eta)$ can be expanded in
a Taylor series around zero as
%
\[
	\mathcal{K}(\eta) = \sum_{j=0}^\infty \mathcal{K}_j(\eta)
	, \qquad
	\mathcal{K}_j(\eta) = D^{-1}G_j(\eta)D^{-1}
	.
\]
%
In particular, note that $\mathcal{K}_0 = \frac{\tanh D}{D}$. In non-dimensional variables, we write the operator with the integral kernel $K_{h_0}$ as $K = \sqrt{\frac{\tanh D}{D}}$, so that we have $\mathcal{K}_0 = K^2$. The Hamiltonian is then expressed as 
%
\begin{equation}
	H  = \frac{1}{2} \int_{\mathbb{R}} \left[ \eta^2 + u \mathcal{K}(\eta) u \right] \, dx.
\label{scaled-Hamiltonian}
\end{equation}
%
The water-wave problem can then be written as a Hamiltonian
system using the variational derivatives of $H$ and posing the
Hamiltonian equations
%
\begin{equation}
	\eta_t = - \partial_x \frac{\delta H}{\delta u}, \qquad u_t = -\partial_x \frac{\delta H}{\delta \eta}.
\end{equation}
%
This system is not in canonical form as the associated structure
map $J_{\eta, u}$ is symmetric:
%
\begin{equation}
	J_{\eta, u} =
	\begin{pmatrix}
		0 & -\partial_x
		\\
		-\partial_x & 0
	\end{pmatrix}.
\end{equation}
%
We now proceed to derive a system of equations which is similar to
the Whitham equation \eqref{whitham-equation}, but allows bi-directional wave propagation. This system will be a stepping stone on the way to a derivation
of \eqref{whitham-equation}, but may also be of independent interest. Consider a wave-
field having a characteristic wavelength $l$ and a characteristic amplitude $a$. Taking into account the nondimensionalization, the two
scalar parameters $\lambda = l / h_0$ and $ \alpha = a / h_0$ appear. In order to introduce the long-wave and small amplitude approximation into the
non-dimensional problem, we use the scaling $\tilde{x}= \frac{1}{\lambda} x$, and $ \eta = \alpha \tilde{\eta}$.
This induces the transformation $\tilde{D} = \lambda D = -\lambda i \partial_x $. If the energy
is nondimensionalized in accord with the nondimensionalization
mentioned earlier, then the natural scaling for the Hamiltonian is
$ \tilde{H}= \alpha^2 H$. In addition, the unknown $u$ is scaled as $u = \alpha \tilde{u}$. The
scaled Hamiltonian \eqref{scaled-Hamiltonian} is then written as
