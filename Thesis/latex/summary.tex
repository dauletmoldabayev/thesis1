%%%%%%%%%%%%%%%%%%%%%%%%%%%%%%%%%%%%%%%%%%%%%%%%%%%%%%%%%%%%%%%%%%%%%%%%%%%%%%%%%%%%%%%%%%%
\chapter{Summary of results}
%%%%%%%%%%%%%%%%%%%%%%%%%%%%%%%%%%%%%%%%%%%%%%%%%%%%%%%%%%%%%%%%%%%%%%%%%%%%%%%%%%%%%%%%%%%

This chapter provides an overview of the results achieved in the course of research work. 
Detailed 

%%%%%%%%%%%%%%%%%%%%%%%%%%%%%%%%%%%%%%%%%%%%%%%%%%%%%%%%%%%%%%%%%%%%%%%%%%%%%%%%%%%%%%%%%%%
\section{The Whitham equation as a model for surface water waves}
%%%%%%%%%%%%%%%%%%%%%%%%%%%%%%%%%%%%%%%%%%%%%%%%%%%%%%%%%%%%%%%%%%%%%%%%%%%%%%%%%%%%%%%%%%%

The Whitham equation was proposed as an alternate model equation for the simplified description of unidirectional 
wave motion at the surface of an inviscid fluid. As the Whitham equation incorporates the full
linear dispersion relation of the water wave problem, it is thought to provide a more faithful description
of shorter waves of small amplitude than traditional long wave models such as the KdV equation.
In this work, we identify a scaling regime in which the Whitham equation can be derived from the
Hamiltonian theory of surface water waves. A Hamiltonian system of Whitham type allowing for two-
way wave propagation is also derived. The Whitham equation is integrated numerically, and it is shown
that the equation gives a close approximation of inviscid free surface dynamics as described by the
Euler equations. The performance of the Whitham equation as a model for free surface dynamics is also
compared to different free surface models: the KdV equation, the BBM equation, and the Padé (2,2) model.
It is found that in a wide parameter range of amplitudes and wavelengths, the Whitham equation performs
on par with or better than the three considered models.


In its simplest form, the water-wave problem concerns the
flow of an incompressible inviscid fluid with a free surface over
a horizontal impenetrable bed. In this situation, the fluid flow
is described by the Euler equations with appropriate boundary
conditions, and the dynamics of the free surface are of particular
interest in the solution of this problem.
There are a number of model equations which allow the
approximate description of the evolution of the free surface
without having to provide a complete solution of the fluid flow
below the surface. In the present contribution, interest is focused

on the derivation and evaluation of a nonlocal water-wave model
known as the Whitham equation. The equation is written as

\begin{equation}
\eta_t + \frac{3}{2} \frac{c_0}{h_0} \eta \eta_x + K_{h_0} \ast \eta_x = 0, 
\label{whitham-equation}
\end{equation}

where the convolution kernel $K_{h_0}$ is given in terms of the Fourier
transform by

\begin{equation}
\mathcal{F}K_{h_0} (\xi) =  \sqrt{\frac{g \tanh(h_0\xi)}{\xi}}.
\label{whitham-kernel}
\end{equation}

$g$ is the gravitational acceleration, $h_0$ is the undisturbed depth of
the fluid, and $c_0 = \sqrt{gh_0}$ is the corresponding long-wave speed.
The convolution can be thought of as a Fourier multiplier operator,
and \eqref{whitham-kernel} represents the Fourier symbol of the operator.
The Whitham equation was proposed by Whitham \cite{Whitham1967} as an
alternative to the well known Korteweg–de Vries (KdV) equation


%%%%%%%%%%%%%%%%%%%%%%%%%%%%%%%%%%%%%%%%%%%%%%%%%%%%%%%%%%%%%%%%%%%%%%%%%%%%%%%%%%%%%%%%%%%
\section{The Whitham equation with surface tension}
%%%%%%%%%%%%%%%%%%%%%%%%%%%%%%%%%%%%%%%%%%%%%%%%%%%%%%%%%%%%%%%%%%%%%%%%%%%%%%%%%%%%%%%%%%%






%%%%%%%%%%%%%%%%%%%%%%%%%%%%%%%%%%%%%%%%%%%%%%%%%%%%%%%%%%%%%%%%%%%%%%%%%%%%%%%%%%%%%%%%%%%
\section{A numerical study of nonlinear dispersive wave models with SpecTraVVave}
%%%%%%%%%%%%%%%%%%%%%%%%%%%%%%%%%%%%%%%%%%%%%%%%%%%%%%%%%%%%%%%%%%%%%%%%%%%%%%%%%%%%%%%%%%%





%%%%%%%%%%%%%%%%%%%%%%%%%%%%%%%%%%%%%%%%%%%%%%%%%%%%%%%%%%%%%%%%%%%%%%%%%%%%%%%%%%%%%%%%%%%
\section{Explicit solutions for a long-wave model with constant vorticity}
%%%%%%%%%%%%%%%%%%%%%%%%%%%%%%%%%%%%%%%%%%%%%%%%%%%%%%%%%%%%%%%%%%%%%%%%%%%%%%%%%%%%%%%%%%%
