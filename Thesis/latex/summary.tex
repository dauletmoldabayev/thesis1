%%%%%%%%%%%%%%%%%%%%%%%%%%%%%%%%%%%%%%%%%%%%%%%%%%%%%%%%%%%%%%%%%%%%%%%%%%%%%%%%%%%%%%%%%%%
\chapter{Summary of results}
%%%%%%%%%%%%%%%%%%%%%%%%%%%%%%%%%%%%%%%%%%%%%%%%%%%%%%%%%%%%%%%%%%%%%%%%%%%%%%%%%%%%%%%%%%%

This chapter provides an overview of the results achieved in the course of research work. 
Detailed 

%%%%%%%%%%%%%%%%%%%%%%%%%%%%%%%%%%%%%%%%%%%%%%%%%%%%%%%%%%%%%%%%%%%%%%%%%%%%%%%%%%%%%%%%%%%
\section{The Whitham equation as a model for surface water waves}
%%%%%%%%%%%%%%%%%%%%%%%%%%%%%%%%%%%%%%%%%%%%%%%%%%%%%%%%%%%%%%%%%%%%%%%%%%%%%%%%%%%%%%%%%%%

The Whitham equation was proposed as an alternate model equation for the simplified description of unidirectional 
wave motion at the surface of an inviscid fluid. As the Whitham equation incorporates the full
linear dispersion relation of the water wave problem, it is thought to provide a more faithful description
of shorter waves of small amplitude than traditional long wave models such as the KdV equation.
In this work, we identify a scaling regime in which the Whitham equation can be derived from the
Hamiltonian theory of surface water waves. A Hamiltonian system of Whitham type allowing for two-
way wave propagation is also derived. The Whitham equation is integrated numerically, and it is shown
that the equation gives a close approximation of inviscid free surface dynamics as described by the
Euler equations. The performance of the Whitham equation as a model for free surface dynamics is also
compared to different free surface models: the KdV equation, the BBM equation, and the Pad\'e (2,2) model.
It is found that in a wide parameter range of amplitudes and wavelengths, the Whitham equation performs
on par with or better than the three considered models.


In its simplest form, the water-wave problem concerns the
flow of an incompressible inviscid fluid with a free surface over
a horizontal impenetrable bed. In this situation, the fluid flow
is described by the Euler equations with appropriate boundary
conditions, and the dynamics of the free surface are of particular
interest in the solution of this problem.
There are a number of model equations which allow the
approximate description of the evolution of the free surface
without having to provide a complete solution of the fluid flow
below the surface. In the present contribution, interest is focused

on the derivation and evaluation of a nonlocal water-wave model
known as the Whitham equation. The equation is written as

\begin{equation}
\eta_t + \frac{3}{2} \frac{c_0}{h_0} \eta \eta_x + K_{h_0} \ast \eta_x = 0, 
\label{whitham-equation}
\end{equation}
where the convolution kernel $K_{h_0}$ is given in terms of the Fourier
transform by

\begin{equation}
\mathcal{F}K_{h_0} (\xi) =  \sqrt{\frac{g \tanh(h_0\xi)}{\xi}}.
\label{whitham-kernel}
\end{equation}
$g$ is the gravitational acceleration, $h_0$ is the undisturbed depth of
the fluid, and $c_0 = \sqrt{gh_0}$ is the corresponding long-wave speed.
The convolution can be thought of as a Fourier multiplier operator,
and \eqref{whitham-kernel} represents the Fourier symbol of the operator.
The Whitham equation was proposed by Whitham \cite{Whitham1967} as an
alternative to the well known Korteweg–de Vries (KdV) equation

\begin{equation}
\eta_t + c_0 \eta_x + \frac{3}{2} \frac{c_0}{h_0} \eta \eta_x + \frac{1}{6} c_0 h_0^2 \eta_{xxx} = 0.
\label{kdv-equation}
\end{equation}


The validity of the KdV equation as a model for surface water waves
can be described as follows. Suppose a wave field with a prominent
amplitude $a$ and characteristic wavelength $l$ is to be studied. The
KdV equation is known to produce a good approximation of the
evolution of the waves if the amplitude of the waves is small and
the wavelength is large when compared to the undisturbed depth,
and if in addition, the two non-dimensional quantities $a / h_0$ and
$h_0^2 / l^2$ are of similar size. The latter requirement can be written in
terms of the Stokes number as

\begin{equation}
\mathcal{S} = \frac{a l^2}{h_0^3} \sim 1. 
\label{stockes-number}
\end{equation}
While the KdV equation is a good model for surface waves if $\mathcal{S} \sim 1$, 
one notorious problem with the KdV equation is that it does
not model accurately the dynamics of shorter waves. Recognizing
this shortcoming of the KdV equation, Whitham proposed to use
the same nonlinearity as the KdV equation, but couple it with a
linear term which mimics the linear dispersion relation of the full
water-wave problem. Thus, at least in theory, the Whitham
equation can be expected to yield a description of the dynamics
of shorter waves which is closer to the solutions of the more
fundamental Euler equations which govern the flow.

The Whitham equation has been studied from a number of
vantage points during recent years. In particular, the existence of
traveling and solitary waves has been established in \citep{Ehrnstrom2012, Ehrnstrom2009}.
Well
posedness of a similar equation was investigated in \citetext{\citealp[]{Lannes2013}-\citealp[]{Klein2015}}, and a
model with variable depth has been studied numerically in \cite{Minzoni2013}.
Moreover, it has been shown in \citep{Hur2015, Sanford2014} that periodic solutions of
the Whitham equation feature modulational instability for short
enough waves in a similar way as small-amplitude periodic wave
solutions of the water-wave problem. However, even though the
equation is routinely mentioned in texts on nonlinear waves \citep{Drazin89, Whitham74}, it appears that the performance of the Whitham equation in
the description of surface water waves has not been investigated
so far.
The purpose of the present article is to give an asymptotic
derivation of the Whitham equation as a model for surface water
waves, and to confirm Whitham’s expectation that the equation is
a fair model for the description of time-dependent surface water
waves. For the purpose of the derivation, we introduce an expo-
nential scaling regime in which the Whitham equation can be de-
rived asymptotically from an approximate Hamiltonian principle
for surface water waves. In order to motivate the use of this scaling,
note that the KdV equation has the property that wide classes of
initial data decompose into a number of solitary waves and small-
amplitude dispersive residue \cite{Ablowitz81}. For the KdV equations, solitary-
wave solutions are known in closed form, and are given by
\begin{equation}
\eta = \frac{a}{h_0}\mbox{sech}^2\left(\sqrt{\frac{3a}{4h_0^3}(x - ct)}\right)
\label{kdv-soliton}
\end{equation}
for a certain wave celerity $c$. These waves clearly comply with
the amplitude–wavelength relation $a / h_0 \sim h_0^2 / l^2$ which was
mentioned above. It appears that the Whitham equation – as indeed do many other nonlinear dispersive equations – also has
the property that broad classes of initial data rapidly decompose
into ordered trains of solitary waves (see Fig. 1). Quantifying the
amplitude–wavelength relation for these solitary waves yields an
asymptotic regime which is expected to be relevant to the validity
of the Whitham equation as a water wave model.

As the curve fit in the right panel of Fig. 1 shows, the relationship
between wavelength and amplitude of the Whitham solitary
waves can be approximately described by the relation $a/h_0 \sim e^{\kappa(l/h_0)^{\nu}}$ for certain values of $\kappa$ and $\nu$. Since the Whitham solitary
waves are not known in exact form, the values of $\kappa$ and $\nu$ have to be
found numerically. Then one may define a Whitham scaling regime
\begin{equation}
\mathcal{W}(\kappa, \nu) = \frac{a}{h_0}e^{\kappa(l/h_0)^{\nu}} \sim 1,
\label{whitham-regime}
\end{equation}
and it will be shown in Sections 2 and 3 that this scaling can be used
advantageously in the derivation of the Whitham equation. The
derivation proceeds by examining the Hamiltonian formulation of
the water-wave problem due to Zakharov, Craig and Sulem [13,14],
and by restricting to wave motion which is predominantly in the
direction of increasing values of x. The approach is similar to the
method of [15], but relies on the new relation (5).
First, in Section 2, a Whitham system is derived which allows for
two-way propagation of waves. The Whitham equation is found in
Section 3. Finally, in Section 4, a comparison of modeling properties
of the KdV and Whitham equations is given. The comparison also
includes the regularized long-wave equation
\begin{equation}
\eta_t + c_0 \eta_x + \frac{3}{2} \frac{c_0}{h_0} \eta \eta_x - \frac{1}{6} h_0^2 \eta_{xxt} = 0,
\label{regularized-long-wave}
\end{equation}
which was put forward in [16] and studied in depth in [17], and
which is also known as the BBM or PBBM equation. The linearized
dispersion relation of this equation is not an exact match to the
dispersion relation of the full water-wave problem, but it is much
closer than the KdV equation, and it might also be expected that
this equation may be able to model shorter waves more success-
fully than the KdV equation. In order to obtain an even better match
of the linear dispersion relation, one may make use of Pad\'e  expansions. In the context of simplified evolution equations, this
approach was pioneered in [18]. For uni-directional models, this
approach was advocated in [19], and in particular, the equation
based on the Pad\'e (2,2) approximation was studied in depth. In dimensional variables, this model takes the form
\begin{equation}
\eta_t + c_0 \eta_x + \frac{3}{2} \frac{c_0}{h_0} \eta \eta_x
- \frac{3}{20} c_0 h_0^2 \eta_{xxx} - \frac{19}{60} h_0^2 \eta_{xxt} = 0.
\label{pade-2-2}
\end{equation}
%%%%%%%%%%%%%%%%%%%%%%%%%%%%%%%%%%%%%%%%%%%%%%%%%%%%%%%%%%%%%%%%%%%%%%%%%%%%%%%%%%%%%%%%%%%
\section{The Whitham equation with surface tension}
%%%%%%%%%%%%%%%%%%%%%%%%%%%%%%%%%%%%%%%%%%%%%%%%%%%%%%%%%%%%%%%%%%%%%%%%%%%%%%%%%%%%%%%%%%%






%%%%%%%%%%%%%%%%%%%%%%%%%%%%%%%%%%%%%%%%%%%%%%%%%%%%%%%%%%%%%%%%%%%%%%%%%%%%%%%%%%%%%%%%%%%
\section{A numerical study of nonlinear dispersive wave models with SpecTraVVave}
%%%%%%%%%%%%%%%%%%%%%%%%%%%%%%%%%%%%%%%%%%%%%%%%%%%%%%%%%%%%%%%%%%%%%%%%%%%%%%%%%%%%%%%%%%%





%%%%%%%%%%%%%%%%%%%%%%%%%%%%%%%%%%%%%%%%%%%%%%%%%%%%%%%%%%%%%%%%%%%%%%%%%%%%%%%%%%%%%%%%%%%
\section{Explicit solutions for a long-wave model with constant vorticity}
%%%%%%%%%%%%%%%%%%%%%%%%%%%%%%%%%%%%%%%%%%%%%%%%%%%%%%%%%%%%%%%%%%%%%%%%%%%%%%%%%%%%%%%%%%%
